\PassOptionsToPackage{unicode=true}{hyperref} % options for packages loaded elsewhere
\PassOptionsToPackage{hyphens}{url}
%
\documentclass[]{article}
\usepackage{hyperref}
\usepackage{lmodern}
\usepackage{amssymb,amsmath}
\usepackage{ifxetex,ifluatex}
\usepackage{fixltx2e} % provides \textsubscript
\ifnum 0\ifxetex 1\fi\ifluatex 1\fi=0 % if pdftex
  \usepackage[T1]{fontenc}
  \usepackage[utf8]{inputenc}
  \usepackage{textcomp} % provides euro and other symbols
\else % if luatex or xelatex
  \usepackage{unicode-math}
  \defaultfontfeatures{Ligatures=TeX,Scale=MatchLowercase}
\fi
% use upquote if available, for straight quotes in verbatim environments
\IfFileExists{upquote.sty}{\usepackage{upquote}}{}
% use microtype if available
\IfFileExists{microtype.sty}{%
\usepackage[]{microtype}
\UseMicrotypeSet[protrusion]{basicmath} % disable protrusion for tt fonts
}{}
\IfFileExists{parskip.sty}{%
\usepackage{parskip}
}{% else
\setlength{\parindent}{0pt}
\setlength{\parskip}{6pt plus 2pt minus 1pt}
}
\usepackage{hyperref}
\hypersetup{
            pdfborder={0 0 0},
            breaklinks=true}
\urlstyle{same}  % don't use monospace font for urls
\setlength{\emergencystretch}{3em}  % prevent overfull lines
\providecommand{\tightlist}{%
  \setlength{\itemsep}{0pt}\setlength{\parskip}{0pt}}
\setcounter{secnumdepth}{0}
% Redefines (sub)paragraphs to behave more like sections
\ifx\paragraph\undefined\else
\let\oldparagraph\paragraph
\renewcommand{\paragraph}[1]{\oldparagraph{#1}\mbox{}}
\fi
\ifx\subparagraph\undefined\else
\let\oldsubparagraph\subparagraph
\renewcommand{\subparagraph}[1]{\oldsubparagraph{#1}\mbox{}}
\fi

% set default figure placement to htbp
\makeatletter
\def\fps@figure{htbp}
\makeatother

\author{Michael Zargham and Emanuel Lima \thanks{The Authors would like to thank Joshua Jodesty, Danilo Lessa Bernadineli, Jamsheed Shorish, Matthew Barlin and Christopher Catoya for their contribution to the formalization of cadCAD as systems modeling and simulation language.}}

\date{May 2021}

\title{cadCAD Formal Specification}
\begin{document}
\maketitle{}
\begin{abstract}
Complex Adaptive Dynamics Computer Aided Design (cadCAD) is a language
for encoding Generalized Dynamical Systems (GDS) as computer programs. As a modeling framework cadCAD is based on best practices from control systems engineering; this framework was designed by Michael Zargham, Markus Koch and Matt Barlin in 2018, in order to bypass the need for powerful closed source software such as Matlab/Simulink in their economic systems design work at BlockScience. The first implementation of cadCAD was developed by Joshua Jodesty, in parallel with the framework development. Python was chosen for the initial implementation due to the wide range of numerical computing tools already available in Python open source data science ecosystem. The Python implementation of cadCAD was Open Sourced in partnership with the Commons Stack in 2019 and has since exploded in use, most notably amongst cryptoeconomic systems designers and token engineers. As of May 2021, there are multiple implementations of cadCAD. This specification was written in part to document the unifying principles for multiple cadCAD implementations spanning Python, Javascript and Rust, and in part to serve as the basis for a new julia implementation of cadCAD for which Emanuel Lima is the lead developer. 
\end{abstract}
\section{Overview\label{cadcad-spec}}

Complex Adaptive Dynamics Computer Aided Design (cadCAD) is a language
for encoding Generalized Dynamical Systems (GDS) as computer programs.
This formal specification is a minimalist description of what a cadCAD engine does.

\begin{quote}
A cadCAD engine is a program which takes an object or record \(x_0 \in \mathcal{X}\) and function or method \(h:\mathcal{X}\rightarrow \mathcal{X}\) and produces a sequence of
records \(x_k\) such that \(x_k = h(x_{k-1})\).
\end{quote}

\hypertarget{state}{%
\subsection{State}\label{state}}

The \emph{state} of the system is called \(x\). The initial state is
\(x_0\) and \(x_k\) is the state of the system at timestep \(k\). Each
state is a point in the statespace \(\mathcal{X}\); said another way,
\(\mathcal{X}\) is a class of record objects and each \(x\) is an
instance of that class.

It is common but not required to implement the statespace as a schema
containing a list of keys, and types constraining the values that can be
stored at that key. Following this convention improves interpretability
of the code and associated results. However, alternative implementations
could improve computational performance.

\hypertarget{dynamics}{%
\subsection{Dynamics}\label{dynamics}}

The dynamics are the rules about how the state at timestep \(k\) is
transformed into the state at timestep \(k+1\). In order to encode any
generalized dynamical system the transform
\(h:\mathcal{X}\rightarrow \mathcal{X}\) is broken down into a
\emph{state update pipeline} \[
h = h_{n-1} \circ h_{n-2} \circ \cdots \circ h_1 \circ h_0
\] where each \(h_i: \mathcal{X}\rightarrow \mathcal{X}\) is called a
\emph{partial state update block}. This pipelining allows the dynamics
to be described as a sequence of transformations each from
\(\mathcal{X}\) to \(\mathcal{X}\). The indices \(i\) denote substeps.
Due to this construction there exist intermediate or virtual states
\(x_{(k,i)}\in \mathcal{X}\). While the intermediate states may be
valuable for debugging, they are merely artifacts of the implementation.

Critical to GDS is the notion of separating non-deterministic (strategic
or stochastic) inputs from the determinist consequences of those inputs.
Therefore each individual \(h_i\) may be decomposed as

\[ \begin{array}{c}g_i:\mathcal{X} \rightarrow \mathcal{U}_i\\
f_i:\mathcal{X}\times \mathcal{U}_i \rightarrow \mathcal{X}\end{array}\]

such that

\[ h_i(x)= f_i(x, g_i(x)). \]

We sometimes call \(g_i\) the blackbox operator and \(f_i\) the whitebox
operator.

\hypertarget{trajectories}{%
\subsection{Trajectories}\label{trajectories}}

While the input to a cadCAD program is an initial state
\(x_0\in\mathcal{X}\) and state update pipeline
\(h:\mathcal{X} \rightarrow \mathcal{X}\) the output of a cadCAD program
is a dataset, that data set is a sequence of records \(x_k\) conforming
to the schema \(\mathcal{X}\), which satisfies the relation \[
x_k = h(x_{k-1})
\] starting from \(x_0\). A \emph{Trajectory} is a sequence

\[
x_0, x_1, \ldots, x_k, \ldots
\]

generated by this process. The length of the sequence is a priori
unbounded. A practical implementation must either specify an exit
condition, the simplest of which is a maximum timestep at which the
iteration will terminate.

In the case where the statespace \(\mathcal{X}\) includes keys and types
for the values associated with those keys one can cast the trajectory as
a table in a database where the record index is the timestep \(k\) and
the columns are the keys in the statespace. This a common approach as it
lends itself well to data analysis once the results are produced.

\hypertarget{termination}{%
\subsection{Termination Conditions}\label{termination}}

Consider the truncated trajectory $\vec X_k= x_0,x_1, \ldots, x_{k-1} \in \mathcal{X}^{k}$ produced by $(x_0, h)$ where $h:\mathcal{X}\rightarrow \mathcal{X}$ and $x_0\in\mathcal{X}$. In order to complete a trajectory and return results it is necessary to define a termination condition $s$, such that when $s$ returns True, the iteration is terminated.

$$ s: \mathcal{X} \times \mathcal{X} \times \cdots \rightarrow {\hbox{\{True, False\}}} $$

where the stopping map $s(\vec X_k)$ is determines whether another iteration is computed. The stopping map can be decomposed into two components

Maximum Iterations: $K$, for $k=|\vec X_k|$ $$(k < K)\in {\hbox{\{True, False\}}}$$
Absorbing States: $A:\mathcal{X} \rightarrow {\hbox{True, False}}$ $$\mathcal{S} =\{x\in \mathcal{X} | A(x) = \hbox{True}\} \subset \mathcal{X}$$
Thus the stopping map is given by $$ s(\vec X_k) = A(x_{k-1}) \lor (k \ge K) $$

computation is terminated at step $k$ if and only if $s(\vec X_k)=$True.

As such, a GDS simulation, or job, is a 3-tuple composed of an initial state, a state update pipeline (that is, the composition of a blackbox and a whitebox operator) and a termination condition.

\hypertarget{sweep}{%
\subsection{Parameter Sweep}\label{sweep}}

When running an experiment, it can be desirable to parameterize the dynamics of the simulation in order to understand the changes that different parameters produce on the resulting trajectories. There are three components of the system model that should be parameterizable: the state, the state update functions and the policies. The expected behavior for an engine implementation would be to set simulations based on the parameters and run these simulations (possibly in parallel, see Implementation Observations). Thus, parameter sweeping is running a different simulation for each of $M$ unique configurations constructed from a Cartesian product of the parameters used to characterize the state, state update functions and policies.

Consider an instance of the GDS job $J=(x_0,h,s)$ which produces a bounded trajectory $\vec X_k$ of length $k\le K$ and stays within the region $\mathcal{X} \setminus \mathcal{S}$. Suppose there is a family of related jobs $\mathcal{J},$ with each job parameterized by a configuration $$ z = {z_1, \ldots z_m } \in \mathcal{Z}_1 \times \cdots \times \mathcal{Z}_m $$

where $\mathcal{Z}_i$ is the domain for parameter $i = 1, \ldots, m$, and $m$ is the total number of different parameter types. A typical job $J \in \mathcal{J}$ would then be configured by the parameter vector $z$, i.e. one may write $J=\hbox{config}(z) \in \mathcal{J}$.

A parameter sweep $\mathcal{P}\subset \mathcal{J}$ is a finite set of jobs $J=(x_0,h,s)$ constructed by specificing a set of cardinality $n_i$ for each parameter $z_i \in \mathcal{Z}i^{n_i}$, which results in $M=\prod{i=1}^m n_i$ distinct jobs indexed $j\in{1,\ldots M}$ and constructed as $$ J_j = \hbox{config}(z_j). $$ The parameter sweep is the set of jobs $$ \mathcal{P} = {\hbox{config}(z) | z \in \mathcal{Z}^{n_1}_1 \times \cdots \times \mathcal{Z}^{n_m}_m \subset \mathcal{Z}_1 \times \cdots \times \mathcal{Z}_m } $$ with finite cardinality $M = |\mathcal{P}|$.

\hypertarget{considerations}{%
\subsection{Considerations for Computer Aided Design}\label{considerations}}

Results from multiple trajectories can be compared side-by-side as long
as experiment meta-data is maintained. As cadCAD is intended for
computation science, additional features for executing repeated
experiments are recommended. Useful controlled repeated experiments
include:

\begin{itemize}
\item Monte Carlo experiments for models with stochastic processes
\item including managing seeds to ensure reproducibility
\item repeated experiments with the same dynamics but different initial state to test sensitivity to initial conditions
\item parametrizing the dynamics by introducing explicit parameters and testing sensitivity to those
parameters
\item modularization of the decomposed dynamics such that a ``\(g_i\)'' block can be swapped out to account for uncertainty in the modelers assumptions
\item modularization of the decomposed dynamics such that ``\(f_i\)'' block can be swapped out to account for a mechanism design change
\item read and write APIs allowing the cadCAD engine runtime to interact with other environments
\end{itemize}

\begin{thebibliography}{widest entry}
 \bibitem[cadCAD]{cadCADpython} J. Jodesty et al (2018-present), \textit{cadCAD Reference Implementation}; \underline{\href{https://github.com/cadCAD-org/cadCAD}{https://github.com/cadCAD-org/cadCAD}}
 \bibitem[intro]{zargham-rice-2019} M. Zargham \& C. Rice (2019), \textit{Introducing Complex Adaptive Dynamics Computer Aided Design}; \underline{\href{https://medium.com/block-science/introducing-complex-adaptive-dynamics-computer-aided-design-cadcad-38b63b541eb8}{BlockScience Medium}}
 \bibitem[tegg]{emmett-2019} J. Emmett (2019), \textit{Official cadCAD Open Source Announcement}; \underline{\href{https://www.youtube.com/watch?v=tFcSqKXfCuE&t=30589s}{Token Engineering Youtube}}
\end{thebibliography}
\end{document}